\documentclass[12pt]{article}
\usepackage[utf8]{inputenc}
\usepackage[spanish]{babel}
\usepackage{amsmath, amssymb, amsthm, amsfonts, apacite, listings}
\usepackage[svgnames]{xcolor}
\usepackage[letterpaper, margin=1in]{geometry}

\lstset{language=R,
  basicstyle=\small\ttfamily,
  stringstyle=\color{DarkGreen},
  otherkeywords={0,1,2,3,4,5,6,7,8,9},
  morekeywords={TRUE,FALSE},
  deletekeywords={data,frame,length,as,character},
  keywordstyle=\color{blue},
  commentstyle=\color{DarkGreen},
  tabsize=2,
}

\title{
  \large{\textbf{Tarea 1\\ Modelamiento Estocástico y Simulación}} \\
	\normalsize{Universidad Técnica Federico Santa María\\ Departamento de Informática\\}
}

\author{
  \normalsize Daniel San Martín Reyes \\ 
  \normalsize \texttt{<daniel.sanmartinr@sansano.usm.cl>}
}

\date{28 de Marzo de 2018}

\begin{document}

  \maketitle
      
  \section*{Tarea}
    Usando el método de Monte Carlo
    
    \begin{enumerate}
      \item Proponer un generador de Números aleatorios provenientes de la $\mathcal{U}(0,1)$ 
        justifique como determina la calidad del generador.

        Para el desarrollo de este trabajo se propone utilizar una combinación de algoritmos
        congruenciales. La combinación de algoritmos congruenciales se basa en
        los siguientes resultados:

        \begin{itemize}
          \item Si $U_1, ..., U_k$ son variables aleatorias iid $\mathcal{U}(0,1)$ la parte
            fraccionaria de $U_1 + ... + U_k$ también sigue una distribución $\mathcal{U}(0,1)$,
            es decir, 
            \begin{equation}
              U_1 + U_2 + ... + U_k - [U_1 + U_2 + ... + U_k] \sim \mathcal{U}(0,1).
            \end{equation}

            Notar que $U_1 + U_2 + ... + U_k - [U_1 + U_2 + ... + U_k]$ es equivalente a 
            $(U_1 + U_2 + ... + U_k) ~ \textrm{mod} ~ 1$

          \item Si $u_1, u_2, ..., u_k$ están generados por algoritmos congruenciales con ciclos 
            de periodo $c_1, c_2, ..., c_k$, respectivamente, entonces la parte fraccionaria de 
            $u_1+u_2+...+u_k$ tiene un ciclo de periodo $\textrm{m.c.m.}\{c_1, c_2, ..., c_k\}$.
        \end{itemize}

        En este trabajo utilizamos el algoritmo combinado de Wichmann y Hill \cite{wichmann1982algorithm}.
        El generador de este algoritmo es:
        
        \begin{equation}
          \begin{split}
            x_i & = 171 ~ x_{i-1} ~ \textrm{mod} ~ 30269 \\
            y_i & = 172 ~ y_{i-1} ~ \textrm{mod} ~ 30307 \\
            z_i & = 170 ~ z_{i-1} ~ \textrm{mod} ~ 30323,
          \end{split}          
        \end{equation}

        luego 

        \begin{equation}
          u_i = \left(\frac{x_i}{30269} + \frac{y_i}{30307} + \frac{z_i}{30323}\right) ~ \textrm{mod} ~ 1.
        \end{equation}

        Para intentar mejorar la implementación, las semillas del algoritmo fueron escogidas utilizando
        la parte decimal de la frecuencia del procesador del equipo en el momento que se ejecuta el código. \medskip

        La determinación de la calidad del generador podría ser realizado mediante el Test $\chi^2$ y el 
        Contraste de Kolmogorov-Smirnov.

        \begin{itemize}
          \item Test $\chi^2$: Dada una muestra $X_1, X_2, ..., X_n$ de una $F_x(x)$ desconocida, se desea
            contrastar
            \begin{equation}
              H_0: F_x(x) = F_0(x) \qquad  H_1:F_x(x) \neq F_0(x).
            \end{equation}
            Realizando una partición del soporte de $X$ en $k$ subconjuntos disjuntos $I_1, I_2, ..., I_k$
            tal que $\bigcup I_i = X$ y $I_i \bigcap i_j = \emptyset$. Luego, se considera el estadístico
            \begin{equation}
              T = \sum_{i=1}^k \frac{(f_i - e_i)^2}{e_i} ~ \sim ~ \chi^2_{(k-1)},
            \end{equation}

            donde $f_i$ corresponde a la frecuncia absoluta del conjunto $i$-ésimo ($I_i$) y $e_i$ es el 
            número de observaciones esperadas en $I_i$ bajo $H_0$. Este test considera la aleatoriedad
            de $F_0 = \mathcal{U}(0,1)$. \medskip

            El problema de este test de bondad de ajuste es que nos permite justificar el rechazo de 
            la hipótesis, pero no ayuda mucho en la aceptación de esta.

          \item Contraste de Kolmogorov-Smirnov: Dada una muestra aleatoria simple $x_1, ..., x_n$, la 
            función de distribución empírica de la muestra es
            \begin{equation}
              F_n(x) = \frac1n \sum_{i=1}^n I_{x_i \leq x} = \frac{|\textrm{valores} ~ x_i \leq x|}{n}.
            \end{equation}

            Las hipótesis son:
            \begin{equation}
              H_0: F_n = F_0 \qquad H_1: F_n \neq F_0, 
            \end{equation}

            donde $F_0$ es la hipotética función de distribución. El estadístico de contraste para el 
            test K-S es:
            \begin{equation}
              D_n = \sup_{x\in\mathbb{R}} |F_n(x) - F_0(x)|.
            \end{equation}

            La distribución exacata de $D_n$ está tabulada para valores de $n \leq 40$ y distintos 
            niveles de significación $\alpha$. Para muestras grandes se utiliza la deistribución 
            asintóticas de $D_n$ dada por 
            \begin{equation}
              \lim_{n\to \infty} P(\sqrt{n} D_n \leq z) = L(z)=1-2\sum_{i=1}^{\infty}(-1)^{i-1}e^{-2i^2z}.
            \end{equation}

            Para el caso particular de que $F_0$ es la función de distribución de una v.a $\mathcal{U}(0,1)$,
            por lo tanto se puede comprobar que el estadístico de Kolmogorov-Smirnov para contraster la 
            uniformidad de la muestra $u_1, ..., u_n$ viene dado por:
            \begin{equation}
              D_n = \max_{1\leq i \leq n} \Big\{\max \Big\{ \Big| \frac{i}{n} - u_{(i)} \Big|, \Big| 
                \frac{i-1}{n} - u_{(i)} \Big| \Big\} \Big\}
            \end{equation}

            siendo $u_{(i)}$ el $i$-ésimo menor valor de la muestra.
        \end{itemize}

        Utilizando las técnicas mencionadas anteriormente obtenemos que el generador propuesto sería al menos útil, 
        porque los resultados nos permiten concluir que no existe evidencia suficiente para rechazar la hipótesis 
        nula, que en este caso particular considera la aleatoriedad de $F_0=\mathcal{U}(0,1)$ dada una muestra de
        números aleatorios creados con nuestro generador. La implementación puede ser revisada en el código
        adjunto.
        
          
      \item Usando los resultados del problema 1 evaluar las integrales 
        \begin{itemize}
          \item $\displaystyle \int_{-\infty}^{\infty}e^{-x^2}dx$
            \begin{equation}
                \theta  = \int_{-\infty}^{\infty}e^{-x^2}dx.
            \end{equation}
            
            Dado que $e^{-x^2}$ es simétrica con respecto a $x=0$, entonces
            \begin{equation}
                \theta  = 2\int_{0}^{\infty}e^{-x^2}dx.
            \end{equation}
            Sea $y = \frac{1}{1+x}$, $dy = -\frac{dx}{(1+x)^2}= -y^2dx$, entonces
            \begin{equation}
                \theta = 2\int_{0}^{1}\frac{e^{-(\frac1y-1)^2}}{y^2}dy
            \end{equation}
            
            De esta forma, podemos estimar $\theta$ como $\hat{\theta} = \mathbb{E}[h(y)]$
            con $ h(y) = 2\frac{e^{-(\frac1y-1)^2}}{y^2}$.
            
            Ejecutando el código adjunto obtenemos $\hat{\theta} \approx 1.77$. \bigskip                        
              
          \item $\displaystyle \int_{0}^{1}\int_{0}^{1}e^{(x+y)^2}dxdy$
            \begin{equation}
                \theta = \int_{0}^{1}\int_{0}^{1}e^{(x+y)^2}dxdy
            \end{equation}
            En este caso, $\theta$ puede ser estimado por $\hat{\theta} = \mathbb{E}[g(u_1, u_2)]$,
            con $U_1, U_2$ v.a.i.i.d $\mathcal{U}(0,1)$, por lo tanto
            \begin{equation}
                \hat{\theta} = \frac1k \sum_{i=1}^ke^{(x+y)^2},
            \end{equation}
            donde $X, Y \sim U(0,1)$.
            
            Ejecutando el código adjunto obtenemos $\hat{\theta} \approx 4.89$.                        
            
        \end{itemize}
      \item  Defina su Problema de simulación y su impacto, Conceptualización
        Modelo, los elementos constructivos, constructos del modelo, forma,
        Recolección de Datos; Construcción del Modelo, Verificación y Validación,
        forma de conducción de los experimentos y Análisis de los Resultados.

        \begin{itemize}
          \item \textbf{Problema:} Modelar la propagación de incendios forestales para ayudar a mejorar 
            procesos de combate y control del fenómeno. En Chile este problema es de alto impacto debido al 
            efecto dañino que tiene a nivel ambiental, económico y social. Es por esto que el desarrollo de 
            modelos de buena calidad, coherentes con la realidad, se hacen importantes para el estudio, análisis 
            y predicción del fenómeno. El principal objetivo es implementar un modelo que ayude a profesionales 
            de instituciones que trabajen con este problema ya sea CONAF, ONEMI, etc.
          \item \textbf{Conceptualización del Modelo:} Para el desarrollo del modelo se busca establecer la 
            relación entre el comportamiento del fuego con la interacción entre el combustible y las condiciones
            meteorológicas del escenario a simular.
          \item \textbf{Recolección de Datos:} Los datos necesarios para el desarrollo del trabajo deben ser 
            obtenidos de fuentes gubernamentales, específicamente de instituciones como CONAF para el modelar
            el combustible y la DMC para conocer el modelo relacionado a los datos meteorológicos. En el caso
            que no exista la posibilidad de acceder a esta información, se puede utilizar fuentes de datos abiertas
            como lo son Google Maps API, Open Weather Map API, NOAA, entre otros.
          \item \textbf{Construcción del Modelo:} Dado que el problema es el manejo y/o control de la 
            propagación del fuego en un incendio forestal, la idea es construir un modelo cualitativamente 
            coherente con la realidad. Para iniciar, se busca comenzar con un modelo basado en autómatas 
            celulares dada que la implementación es considerablemente más simple que comenzar con uno basado
            en ecuaciones diferenciales. El modelo final espera ser construido mediante ecuaciones diferenciales
            estocásticas, siempre que el tiempo así lo permita.
          \item \textbf{Verificación y Validación:} Para verificar el modelo se intentará contactar con profesionales
            expertos en el ámbito. Para la validación se espera contar con datos de las instituciones pertinentes
            y que nos permitan contrastar resultados.
          \item \textbf{Conducción de los Experimentos:} Por definir.
          \item \textbf{Análisis de los Resultados:} Por realizar.
        \end{itemize}

    \end{enumerate}
        
    % \begin{enumerate}
    %     \item $\displaystyle \int_{0}^5(1+x^2)^{3/2}dx$
    %         \begin{equation}
    %             \theta = \int_{0}^{5}(1+x^2)^{3/2}dx
    %         \end{equation}
    %         Sea $y = \frac{x}{5}$, $dy = \frac{dx}{5}$, entonces
            
    %         \begin{equation}
    %             \begin{split}
    %                 \theta & = \int_{0}^{5}(1+x^2)^{3/2}dx \\
    %                     & = \int_{0}^{1}5(1+(5y)^2)^{3/2}dy \\
    %                     & = \int_{0}^{1}5(1+25y^2)^{3/2}dy.
    %             \end{split}
    %             %\theta = \int_{0}^{1}5(1+25y^2)^{3/2}dy
    %         \end{equation}
            
    %         De esta forma, podemos estimar $\theta$ como $\hat{\theta} = \mathbb{E}[h(y)]$
    %         con $ h(y) = 5(1+25y^2)^{3/2}$.
        
    %         Ejecutando el código adjunto obtenemos $\hat{\theta} \approx 176.0202$. \bigskip
            
    %     \item $\displaystyle \int_{0}^{\infty}x(1+x^2)^{-2}dx$
    %         \begin{equation}
    %             \theta = \int_{0}^{\infty}x(1+x^2)^{-2}dx \\
    %         \end{equation}
            
    %         Sea $y = \frac{1}{1+x}$, $dy = -\frac{dx}{(1+x)^2}= -y^2dx$, entonces
            
    %         \begin{equation}
    %             \begin{split}
    %                 \theta & = \int_{0}^{\infty}x(1+x^2)^{-2}dx \\
    %                     & = \int_{1}^{0}-\frac{(1/y - 1)(1 + (1/y - 1)^2)^{-2}dy}{y^2} \\
    %                     & = \int_{0}^{1}\frac{(1/y - 1)}{(y(1 + (1/y - 1)^2))^2}dy
    %             \end{split}
    %         \end{equation}
            
    %         De esta forma, podemos estimar $\theta$ como $\hat{\theta} = \mathbb{E}[h(y)]$
    %         con $ h(y) = \frac{(1/y - 1)}{(y(1 + (1/y - 1)^2))^2}$.
        
    %         Ejecutando el código adjunto obtenemos $\hat{\theta} \approx 0.5003132$. \bigskip
            
    %     \item $\displaystyle \int_{-\infty}^{\infty}e^{-x^2}dx$
    %         % \begin{equation}
    %         %     \begin{split}
    %         %         \theta & = \int_{-\infty}^{\infty}e^{-x^2}dx \\
    %         %          & = \int_{-\infty}^{0}e^{-x^2}dx + \int_{0}^{\infty}e^{-x^2}dx
    %         %     \end{split}
    %         % \end{equation}
            
    %         % Sea $u=\frac{1}{1+x} \implies du=-\frac{dx}{(1+x)^2}=-u^2dx$ y $v=-\frac{1}{1+x} \implies dv=\frac{dx}{(1+x)^2}=v^2dx$, 
    %         % entonces
    %         % \begin{equation}
    %         %     \begin{split}
    %         %         \theta & = \int_{-\infty}^{-1}e^{-x^2}dx + \int_{1}^{0}-\frac{e^{-(1/u-1)^2}}{u^2}du \\
    %         %     \end{split}
    %         % \end{equation}
            
    %         \begin{equation}
    %             \theta  = \int_{-\infty}^{\infty}e^{-x^2}dx.
    %         \end{equation}
            
    %         Dado que $e^{-x^2}$ es simétrica con respecto a $x=0$, entonces
    %         \begin{equation}
    %             \theta  = 2\int_{0}^{\infty}e^{-x^2}dx,
    %         \end{equation}
    %         Sea $y = \frac{1}{1+x}$, $dy = -\frac{dx}{(1+x)^2}= -y^2dx$, entonces
    %         \begin{equation}
    %             \theta = 2\int_{0}^{1}\frac{e^{-(\frac1y-1)^2}}{y^2}dy
    %         \end{equation}
            
    %         De esta forma, podemos estimar $\theta$ como $\hat{\theta} = \mathbb{E}[h(y)]$
    %         con $ h(y) = 2\frac{e^{-(\frac1y-1)^2}}{y^2}$.
            
    %         Ejecutando el código adjunto obtenemos $\hat{\theta} \approx 1.772264$. \bigskip
            
    %     \item $\displaystyle \int_{0}^{1}\int_{0}^{1}e^{-(x+y)^2}dxdy$
    %         \begin{equation}
    %             \int_{0}^{1}\int_{0}^{1}e^{-(x+y)^2}dxdy
    %         \end{equation}
    %         En este caso, $\theta$ puede ser estimado por $\hat{\theta} = \mathbb{E}[g(u_1, u_2)]$,
    %         con $U_1, U_2$ v.a.i.i.d $\mathcal{U}(0,1)$, por lo tanto
    %         \begin{equation}
    %             \hat{\theta} = \frac1k \sum_{i=1}^ke^{-(x+y)^2},
    %         \end{equation}
    %         donde $X, Y \sim U(0,1)$.
            
    %         Ejecutando el código adjunto obtenemos $\hat{\theta} \approx 4.891916$.
    % \end{enumerate}
        
  \nocite{ross2013}
  
  \bibliographystyle{newapa}
  \bibliography{references}
  
  \section*{Código} \label{sec:code}
    \lstinputlisting[language=R]{../code/homework1.R}


\end{document}
