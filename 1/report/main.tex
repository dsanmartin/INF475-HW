\documentclass[12pt]{article}
\usepackage[utf8]{inputenc}
\usepackage[spanish]{babel}
\usepackage{amsmath, amssymb, amsthm, amsfonts, apacite, listings}
\usepackage[svgnames]{xcolor}
\usepackage[letterpaper, margin=1in]{geometry}

\lstset{language=R,
    basicstyle=\small\ttfamily,
    stringstyle=\color{DarkGreen},
    otherkeywords={0,1,2,3,4,5,6,7,8,9},
    morekeywords={TRUE,FALSE},
    deletekeywords={data,frame,length,as,character},
    keywordstyle=\color{blue},
    commentstyle=\color{DarkGreen},
    tabsize=2,
}

\title{
    \large{
        \textbf{Tarea 1\\ Modelamiento Estocástico y Simulación}} \\
			\normalsize{Universidad Técnica Federico Santa María\\ Departamento de Informática\\	}
		}
\author{
    \normalsize Daniel San Martín Reyes \\ 
    \normalsize \texttt{<daniel.sanmartinr@sansano.usm.cl>}}
\date{Marzo del 2018}

\begin{document}

    \maketitle
        
    \section*{Tarea}
        Usando el método de Monte Carlo
        
        \begin{enumerate}
            \item Proponer un generador de Números aleatorios provenientes de la $\mathcal{U}(0,1)$ 
                justifique como determina la calidad del generador.
                
            \item Usando los resultados del problema 1 evaluar las integrales 
                \begin{itemize}
                    \item $\displaystyle \int_{-\infty}^{\infty}e^{-x^2}dx$
                        \begin{equation}
                            \theta  = \int_{-\infty}^{\infty}e^{-x^2}dx.
                        \end{equation}
                        
                        Dado que $e^{-x^2}$ es simétrica con respecto a $x=0$, entonces
                        \begin{equation}
                            \theta  = 2\int_{0}^{\infty}e^{-x^2}dx,
                        \end{equation}
                        Sea $y = \frac{1}{1+x}$, $dy = -\frac{dx}{(1+x)^2}= -y^2dx$, entonces
                        \begin{equation}
                            \theta = 2\int_{0}^{1}\frac{e^{-(\frac1y-1)^2}}{y^2}dy
                        \end{equation}
                        
                        De esta forma, podemos estimar $\theta$ como $\hat{\theta} = \mathbb{E}[h(y)]$
                        con $ h(y) = 2\frac{e^{-(\frac1y-1)^2}}{y^2}$.
                        
                        Ejecutando el código adjunto obtenemos $\hat{\theta} \approx 1.772264$. \bigskip                        
                        
                    \item $\displaystyle \int_{0}^{1}\int_{0}^{1}e^{-(x+y)^2}dxdy$
                        \begin{equation}
                            \int_{0}^{1}\int_{0}^{1}e^{-(x+y)^2}dxdy
                        \end{equation}
                        En este caso, $\theta$ puede ser estimado por $\hat{\theta} = \mathbb{E}[g(u_1, u_2)]$,
                        con $U_1, U_2$ v.a.i.i.d $\mathcal{U}(0,1)$, por lo tanto
                        \begin{equation}
                            \hat{\theta} = \frac1k \sum_{i=1}^ke^{-(x+y)^2},
                        \end{equation}
                        donde $X, Y \sim U(0,1)$.
                        
                        Ejecutando el código adjunto obtenemos $\hat{\theta} \approx 4.891916$.                        
                    
                \end{itemize}
            \item  Defina su Problema de simulación y su impacto , Conceptualización
                Modelo, los elementos constructivos, constructos del modelo, forma
                Recolección de Datos ; Construcción del Modelo, Verificación y Validación,
                forma de conducción delos experimentos y Análisis de los Resultados
        \end{enumerate}
        
        % \begin{enumerate}
        %     \item $\displaystyle \int_{0}^5(1+x^2)^{3/2}dx$
        %         \begin{equation}
        %             \theta = \int_{0}^{5}(1+x^2)^{3/2}dx
        %         \end{equation}
        %         Sea $y = \frac{x}{5}$, $dy = \frac{dx}{5}$, entonces
                
        %         \begin{equation}
        %             \begin{split}
        %                 \theta & = \int_{0}^{5}(1+x^2)^{3/2}dx \\
        %                     & = \int_{0}^{1}5(1+(5y)^2)^{3/2}dy \\
        %                     & = \int_{0}^{1}5(1+25y^2)^{3/2}dy.
        %             \end{split}
        %             %\theta = \int_{0}^{1}5(1+25y^2)^{3/2}dy
        %         \end{equation}
                
        %         De esta forma, podemos estimar $\theta$ como $\hat{\theta} = \mathbb{E}[h(y)]$
        %         con $ h(y) = 5(1+25y^2)^{3/2}$.
            
        %         Ejecutando el código adjunto obtenemos $\hat{\theta} \approx 176.0202$. \bigskip
                
        %     \item $\displaystyle \int_{0}^{\infty}x(1+x^2)^{-2}dx$
        %         \begin{equation}
        %             \theta = \int_{0}^{\infty}x(1+x^2)^{-2}dx \\
        %         \end{equation}
                
        %         Sea $y = \frac{1}{1+x}$, $dy = -\frac{dx}{(1+x)^2}= -y^2dx$, entonces
                
        %         \begin{equation}
        %             \begin{split}
        %                 \theta & = \int_{0}^{\infty}x(1+x^2)^{-2}dx \\
        %                     & = \int_{1}^{0}-\frac{(1/y - 1)(1 + (1/y - 1)^2)^{-2}dy}{y^2} \\
        %                     & = \int_{0}^{1}\frac{(1/y - 1)}{(y(1 + (1/y - 1)^2))^2}dy
        %             \end{split}
        %         \end{equation}
                
        %         De esta forma, podemos estimar $\theta$ como $\hat{\theta} = \mathbb{E}[h(y)]$
        %         con $ h(y) = \frac{(1/y - 1)}{(y(1 + (1/y - 1)^2))^2}$.
            
        %         Ejecutando el código adjunto obtenemos $\hat{\theta} \approx 0.5003132$. \bigskip
                
        %     \item $\displaystyle \int_{-\infty}^{\infty}e^{-x^2}dx$
        %         % \begin{equation}
        %         %     \begin{split}
        %         %         \theta & = \int_{-\infty}^{\infty}e^{-x^2}dx \\
        %         %          & = \int_{-\infty}^{0}e^{-x^2}dx + \int_{0}^{\infty}e^{-x^2}dx
        %         %     \end{split}
        %         % \end{equation}
                
        %         % Sea $u=\frac{1}{1+x} \implies du=-\frac{dx}{(1+x)^2}=-u^2dx$ y $v=-\frac{1}{1+x} \implies dv=\frac{dx}{(1+x)^2}=v^2dx$, 
        %         % entonces
        %         % \begin{equation}
        %         %     \begin{split}
        %         %         \theta & = \int_{-\infty}^{-1}e^{-x^2}dx + \int_{1}^{0}-\frac{e^{-(1/u-1)^2}}{u^2}du \\
        %         %     \end{split}
        %         % \end{equation}
                
        %         \begin{equation}
        %             \theta  = \int_{-\infty}^{\infty}e^{-x^2}dx.
        %         \end{equation}
                
        %         Dado que $e^{-x^2}$ es simétrica con respecto a $x=0$, entonces
        %         \begin{equation}
        %             \theta  = 2\int_{0}^{\infty}e^{-x^2}dx,
        %         \end{equation}
        %         Sea $y = \frac{1}{1+x}$, $dy = -\frac{dx}{(1+x)^2}= -y^2dx$, entonces
        %         \begin{equation}
        %             \theta = 2\int_{0}^{1}\frac{e^{-(\frac1y-1)^2}}{y^2}dy
        %         \end{equation}
                
        %         De esta forma, podemos estimar $\theta$ como $\hat{\theta} = \mathbb{E}[h(y)]$
        %         con $ h(y) = 2\frac{e^{-(\frac1y-1)^2}}{y^2}$.
                
        %         Ejecutando el código adjunto obtenemos $\hat{\theta} \approx 1.772264$. \bigskip
                
        %     \item $\displaystyle \int_{0}^{1}\int_{0}^{1}e^{-(x+y)^2}dxdy$
        %         \begin{equation}
        %             \int_{0}^{1}\int_{0}^{1}e^{-(x+y)^2}dxdy
        %         \end{equation}
        %         En este caso, $\theta$ puede ser estimado por $\hat{\theta} = \mathbb{E}[g(u_1, u_2)]$,
        %         con $U_1, U_2$ v.a.i.i.d $\mathcal{U}(0,1)$, por lo tanto
        %         \begin{equation}
        %             \hat{\theta} = \frac1k \sum_{i=1}^ke^{-(x+y)^2},
        %         \end{equation}
        %         donde $X, Y \sim U(0,1)$.
                
        %         Ejecutando el código adjunto obtenemos $\hat{\theta} \approx 4.891916$.
        % \end{enumerate}
        
    \nocite{book:ross2013}
    
    \bibliographystyle{newapa}
    \bibliography{references}
    
    \section*{Código}
        \lstinputlisting[language=R]{code/homework1.R}

\end{document}
